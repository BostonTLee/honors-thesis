\documentclass{article}

\usepackage[backend=bibtex]{biblatex}

\usepackage[margin=1in]{geometry}

\usepackage{booktabs}

\usepackage{enumitem}

\usepackage{amsmath}
\usepackage{amsthm}
\usepackage{amssymb}

\usepackage[bookmarks, bookmarksopen=true, plainpages=false, pdfpagelabels,
pdfpagelayout=SinglePage, breaklinks = true]{hyperref}

\bibliography{references.bib}

\title{The Relationship Between Potential Financial Stress and Mental Health in the United States}
\author{Boston Lee}

\begin{document}

\maketitle

\section{Introduction}

Given that it may be intuitively plausible that financial hardship has a
delitarious effect on mental health on a personal level,
we aim to answer whether there is any relationship
between financial hardship and mental health on a group level.
If it were discovered that financial hardship and mental
health issues covary, it would perhaps inform
either mental health policy, economic policy, or both.
Although the standard statistical methods used in this paper
cannot be used to draw causal conclusions with observational data,
perhaps insight into whether there is evidence that there
is a linear relationship between poverty and negative mental health effects
could be used to guide more sophisticated future research.

Note that given the unavailability of individual-level data,
and the implausibility of conducting randomized trials with our chosen responses,
some compromises were made in the research questions explored in this paper.
Data for our responses and predictors was only available on a regional level,
hence any inferential conclusions drawn
concern geographic regions, not individuals.
Furthermore, any inferential conclusions drawn will concern correlative
relationships.
Data about prevalence of mental health issues and poverty is observational
in nature, and thus modeling these phenomena does not license causal
conclusions.

\section{Data Sources}

TODO

The data included in the final dataset came from two main sources, the American
Community Survey (ACS), and the Substance Abuse and Mental Health Services
Administration (SAMHSA). The main three response variables originated from the
SAMHSA data, while ACS provided demographic data used as controlling variables.

\subsection{Variables}\label{sec:finalvars}

Note that in all of the below descriptions, the variables are described in their
non-aggregated state. A full description of the census-level variables would
include a caveat that the ACS variables are actually weighted averages of
county-level estimates. For more information on variable aggregation, see
Section~\ref{sec:aggregation}. Treat references to, for instance, the
median income, as references to the population-weighted average of
median income across a given substate region.

It should also be noted that because data is aggregated to regional levels, we
cannot infer a relationship between financial variables and mental health
variables on an individual level, but rather on a regional level. We may be
able to say that, for instance, a difference of $x$ in the average proportion
of individuals living below poverty in a region is correlated to a difference
of $y$ in the average proportion of individuals experiencing a major depressive
episode. We may \textbf{not} say, given our data, that a difference of $x$ in
an individual's average income is correlated to a difference of $y$ in the
average of an individual's probability of experiencing a major depressive
episode.

In addition, all variables are relative to time period.
See Section~\ref{sec:variables} for details.

The main response variables were chosen to measure mental health
symptoms that could have a relationship to environmental factors.
General mental illness, for instance, can be related to genetic factors
even more strongly than our predictor(s) of interest. The
mental-health-related variables chosen were: the proportion of individuals 12
or older who reported experiencing a major depressive episode in the
past year, the proportion of individuals 12 or older who reported
experiencing suicidal thoughts in the past year, and the proportion of
individuals 12 or older who reported an alcohol use disorder in the past
year. Each of these variables could intuitively have a relationship with
environmental variables, notably income, rather than genetic factors.

The main predictor variable, the (population weighted average of) the
proportion of individuals (FIXME) below the poverty level, was chosen
because we are attempting to discern the relationship between financial
stress and mental health. The proportion of individuals below the
poverty level encodes information about the lower tail of the
distribution of incomes, not simply the center of the distribution.
If a given region has more people living below the poverty line, we
would expect that to give us more information about financial stress
than simply a measure of centrality.

A secondary question is whether the (population-weighted average of)
median income of a region does give us more information about mental
health outcomes, even given the proportion of individuals below the
poverty level. It is still feasible that a measure of center does encode
relevant information about financial stress, or is still related to
mental health outcomes. Median income was included rather than mean
income because mean income is sensitive to larger outliers, meaning in a
region where a proportion of the population is very wealthy, mean income
does not carry as much information about the income of a middle-income person
living in the region.

The model also included several demographic variables to control for
variability.

The proportion of population of Caucasian and African
American descent were chosen to control for potential socio-economic
effects on mental health outcomes. Other ethnicities were omitted to
ensure that no multicollinearity (FIXME) occured, given that knowing
multiple demographic variables gives information about what values other
proportions may be.

Median age was included as a controlling variable
both because median age could be related to mental health effects (in
that a region with a younger average population may experience more
reported or actual mental health problems in general) or income effects.

Education was accounted for by including the proportion of the
population in a region with a high school diploma or above. This
variable was included primarily to control for effects on income, given
that greater educational achievement is correlated with greater income.

Finally, the proportion of the population 15 years or older that is
married was included, to control for socioeconomic effects arising from
a greater marriage rate. For instance, reported household income is
likely higher in areas where more households consist of married
individuals (FIXME).

%
% county_fips
% substate_region_id
% substate_region_name
% state_fips
% state
% prop_MDE
% state_fips_x
% prop_suicidal_thoughts
% state_fips_y
% prop_alcohol_use_disorder
% total_pop
% prop_white
% prop_black
% median_age
% median_household_income
% prop_25_years_over_high_school
% prop_married_15_years_and_older
% prop_below_poverty_level
% fips
%



For a full description of variables, including data sources, see
Appendix, Section \ref{sec:variables}.

\subsection{Data Joining}\label{sec:joining}

The dataset was formed with substate regions corresponding to
``substate/aggregate region-level estimates included in the [SAMHSA] maps"
\cite{samhsa_data}. All substate regions appearing in datasets with the response
variables of interest were used as individual data points in the final dataset.
However, the final dataset also included ACS county-level estimates. As such,
the SAMHSA estimates were duplicated for every county belonging to the region,
and then aggregated back to substate-region estimates (see Section
\ref{sec:aggregation}).

\subsection{Data Aggregation}\label{sec:aggregation}

All variables in the model dataset were aggregated to the substate region level,
as defined by the SAMHSA data \cite{samhsa_substate_region_defs}.
The ACS estimates were only available at the county level,
while the response variables were only available at the substate region level.
Hence, while the data was initially represented at the county level,
with repeated observations for the SAMHSA estimates
(see Section \ref{sec:joining}),
the final models were constructed at the substate region level.
The ACS estimates were collapsed to the substate region level
by a population weighted average for the substate region in question.
Each numeric variable was grouped by substate region
in the following way:
Assume group $g$ contains dataset indices $I_g$,
each of which is associated with a county-level population
$p_{i}, i \in I_g$.
Then given a set of observations $x_{i}, i \in I_g$
from a numeric variable,
the condensed estimate $x_g$ would be as follows:

\begin{equation*}
    x_g = \sum_{i \in I_g}
    \left[ \frac{p_{i}}{\sum_{k \in I_g} p_{k}} x_i \right]
\end{equation*}

Note that because of the joining procedure,
we want the SAMHSA variables measured at the substate regional level,
that is, the level to which we are aggregating,
to remain the same as before the joining procedure.
We can see that the above approach accomplishes this.
If all $x_i, i \in I_g$ are equal to $\xi$,
then our procedure reduces to the following:

\begin{equation*}
    x_g =
    \sum_{i \in I_g}
    \left[ \frac{p_{i}}{\sum_{k \in I_g} p_{k}} x_i \right] =
    \sum_{i \in I_g}
    \left[ \frac{p_{i}}{\sum_{k \in I_g} p_{k}} \xi \right] =
    \left[ \frac{\sum_{i \in I_g}p_{i}}{\sum_{k \in I_g} p_{k}} \xi \right] =
    \xi
\end{equation*}

This type of aggregation thus has an asymmetrical (albeit desirable)
effect on the variables from different sources.
It leaves the substate-region-level SAMHSA estimates unchanged,
while providing substate-region-level ACS aggregated estimates
that reflect the weight of relative population within a given substate region.

After the county-level population estimates were used
to create a weighted average for the numeric variables,
they were dropped from the final dataset.
That is, population was not used as a predictor or response variable.
Only the variables mentioned in \ref{sec:finalvars}
were included in the final model.

\section{Descriptive Statistics}
\section{Methods}
\section{Results}
\section{Discussion}

\section{Appendix}

\subsection{Variable Descriptions}\label{sec:variables}

\newcommand\censuscodes{https://www.census.gov/programs-surveys/acs/technical-documentation/code-lists.2018.html}
\newcommand\descentdesc[1]{ Proportion of the population which is of
    #1 descent. See \url{\censuscodes} for more
    pecise definitions.}
\newcommand\descentreason{Ethnicity is tied to socioeconomic status, which
    could in turn be correlated to one's financial status and mental health.
    Variables related to ethnicity were included to control for variation.}
\newcommand\agedescto[2]{Proportion of the population from #1 to #2 years old.}
\newcommand\agedescover[1]{Proportion of the population over #1 years old.}
\newcommand\agereason{Overall economic status could vary with age, so age
    variables were included to control for variation.}
\newcommand\educationdesc[1]{Proportion of the population 25 years of age and
    over whose maximum educational attainment is #1.}
\newcommand\educationreason{Educational attainment can be used as a general
    measure of socioeconomic status. Variables related to educational
    attainment were included to account for variation.}
\newcommand\maritaldesc[1]{Proportion of the population age 15 and older who
    have marital status ``#1".}
\newcommand\maritalreason{Marital status reflects general socioeconomic status.
    Variables related to marital status are included to account for variation.}

\begin{itemize}[label={}, align=left]
    \item[\texttt{county\_fips}] \
          \begin{description}
              \item[Description] FIPS code at the county level (three digits)
              \item[Reason for inclusion] Used as a merge key for all datasets.
                    Extracted from longer FIPS codes as needed. \\
              \item[Data Source]
                    \cite{acs_demographics_data},
                    \cite{acs_poverty_data},
                    \cite{acs_income_data},
                    \cite{acs_marital_data},
                    \cite{acs_education_data},
                    \cite{samhsa_data} \\
              \item[Modifications] County-level FIPS was defined within a
                    larger string. This string was sliced and zero-padded for all
                    datasets to produce a consistent key. \\
          \end{description}
    \item[\texttt{substate\_region\_id}] \
          \begin{description}
              \item[Description] A numeric value that corresponds to the
                    substate region name within a state. If a state has "n"
                    substate regions, then this variable will take on values 1,
                    2, ..., n, within that state.
              \item[Reason for inclusion] This, in combination with state ID,
                    gives a unique identifier for the substate regions used in the
                    SAMHSA data.
              \item[Data Source] \cite{samhsa_substate_region_defs}
              \item[Modifications] N/A
          \end{description}
    \item[\texttt{substate\_region\_name}] \
          \begin{description}
              \item[Description] The substate region name used in the 2014-2016
                    substate report, attached at

                    \parbox[t]{\textwidth}{\citefield{samhsa_substate_region_defs}{url}}

              \item[Reason for inclusion] Human-readable name for substate
                    regions.
              \item[Data Source] \cite{samhsa_substate_region_defs}
              \item[Modifications] N/A
          \end{description}
    \item[\texttt{state\_fips}] \
          \begin{description}
              \item[Description] FIPS code at the state level (two digits) \\
              \item[Reason for inclusion] Used as a merge key for all datasets.
                    Extracted from longer FIPS codes as needed. \\
              \item[Data Source]
                    \cite{acs_demographics_data},
                    \cite{acs_poverty_data},
                    \cite{acs_income_data},
                    \cite{acs_marital_data},
                    \cite{acs_education_data},
                    \cite{samhsa_data} \\
              \item[Modifications] State-level FIPS was defined within a
                    larger string. This string was sliced and zero-padded for all
                    datasets to produce a consistent key. \\
          \end{description}
    \item[\texttt{census\_tract\_code}] \
          \begin{description}
              \item[Description] 2010 Geography census tract code (where
                    applicable).
              \item[Reason for inclusion] Used for identifying tracts belonging
                    to a substate region defined at the census tract level.
              \item[Data Source] \cite{samhsa_substate_region_defs}
              \item[Modifications] N/A
          \end{description}
    \item[\texttt{prop\_MDE}] \
          \begin{description}
              \item[Description] Proportion of adults aged 18 or older major
                    who have had a major depressive episode in the past year
              \item[Reason for inclusion] This is a candidate for a response
                    variable. Having a major depressive episode(s) is more
                    closely linked to life events than general mental illness, in
                    which there is a lot of variation.
              \item[Data Source] \cite{samhsa_data}
              \item[Modifications] Changed from a percentage to a proportion.
          \end{description}
    \item[\texttt{prop\_suidical\_thoughts}] \
          \begin{description}
              \item[Description] Proportion of individuals aged 12 or older
                    with an alcohol use disorder in the past year \\
              \item[Reason for inclusion] This is a potential response variable
                    to measure overall mental health in a population. Suicidal
                    thoughts in particular seem to be more closely linked to life
                    circumstances than genetic factors, as opposed to general
                    mental illness.
              \item[Data Source] \cite{samhsa_data} \\
              \item[Modifications] Changed from a percentage to a proportion.
          \end{description}
    \item[\texttt{prop\_white}] \
          \begin{description}
              \item[Description Auto] \descentdesc{Caucasian}
              \item[Reason for inclusion] \descentreason
              \item[Data Source] \cite{acs_demographics_data}
              \item[Modifications] Changed from a percentage to a proportion.
          \end{description}
    \item[\texttt{prop\_black}] \
          \begin{description}
              \item[Description] \descentdesc{African American}
              \item[Reason for inclusion] \descentreason
              \item[Data Source] \cite{acs_demographics_data}
              \item[Modifications] Changed from a percentage to a proportion.
          \end{description}
    \item[\texttt{prop\_native}] \
          \begin{description}
              \item[Description] \descentdesc{Native}
              \item[Reason for inclusion] \descentreason
              \item[Data Source] \cite{acs_demographics_data}
              \item[Modifications] Changed from a percentage to a proportion.
          \end{description}
    \item[\texttt{prop\_asian}] \
          \begin{description}
              \item[Description] Description of what it does
              \item[Reason for inclusion] \descentreason
              \item[Data Source] \cite{acs_demographics_data}
              \item[Modifications] Changed from a percentage to a proportion.
          \end{description}
    \item[\texttt{prop\_pacific\_islander}] \
          \begin{description}
              \item[Description] \descentdesc{Pacific Islander}
              \item[Reason for inclusion] \descentreason
              \item[Data Source] \cite{acs_demographics_data}
              \item[Modifications] Changed from a percentage to a proportion.
          \end{description}
    \item[\texttt{prop\_latino}] \
          \begin{description}
              \item[Description] \descentdesc{Latino}
              \item[Reason for inclusion] \descentreason
              \item[Data Source] \cite{acs_demographics_data}
              \item[Modifications] Changed from a percentage to a proportion.
          \end{description}
    \item[\texttt{prop\_other\_race}] \
          \begin{description}
              \item[Description] \descentdesc{any not previously included (single-ethnicity)}
              \item[Reason for inclusion] \descentreason
              \item[Data Source] \cite{acs_demographics_data}
              \item[Modifications] Changed from a percentage to a proportion.
          \end{description}
    \item[\texttt{prop\_two\_or\_more\_races}] \
          \begin{description}
              \item[Description] \descentdesc{two or more races in}
              \item[Reason for inclusion] \descentreason
              \item[Data Source] \cite{acs_demographics_data}
              \item[Modifications] Changed from a percentage to a proportion.
          \end{description}
    \item[\texttt{prop\_15\_to\_19\_years}] \
          \begin{description}
              \item[Description] \agedescto{15}{19}
              \item[Reason for inclusion] \agereason
              \item[Data Source] \cite{acs_demographics_data}
              \item[Modifications] Changed from a percentage to a proportion.
          \end{description}
    \item[\texttt{prop\_20\_to\_24\_years}] \
          \begin{description}
              \item[Description] \agedescto{20}{24}
              \item[Reason for inclusion] \agereason
              \item[Data Source] \cite{acs_demographics_data}
              \item[Modifications] Changed from a percentage to a proportion.
          \end{description}
    \item[\texttt{prop\_25\_to\_34\_years}] \
          \begin{description}
              \item[Description] \agedescto{25}{30}
              \item[Reason for inclusion] \agereason
              \item[Data Source] \cite{acs_demographics_data}
              \item[Modifications] Changed from a percentage to a proportion.
          \end{description}
    \item[\texttt{prop\_35\_to\_44\_years}] \
          \begin{description}
              \item[Description] \agedescto{35}{44}
              \item[Reason for inclusion] \agereason
              \item[Data Source] \cite{acs_demographics_data}
              \item[Modifications] Changed from a percentage to a proportion.
          \end{description}
    \item[\texttt{prop\_45\_to\_54\_years}] \
          \begin{description}
              \item[Description] \agedescto{45}{54}
              \item[Reason for inclusion] \agereason
              \item[Data Source] \cite{acs_demographics_data}
              \item[Modifications] Changed from a percentage to a proportion.
          \end{description}
    \item[\texttt{prop\_55\_to\_59\_years}] \
          \begin{description}
              \item[Description] \agedescto{55}{59}
              \item[Reason for inclusion] \agereason
              \item[Data Source] \cite{acs_demographics_data}
              \item[Modifications] Changed from a percentage to a proportion.
          \end{description}
    \item[\texttt{prop\_60\_to\_64\_years}] \
          \begin{description}
              \item[Description] \agedescto{60}{64}
              \item[Reason for inclusion] \agereason
              \item[Data Source] \cite{acs_demographics_data}
              \item[Modifications] Changed from a percentage to a proportion.
          \end{description}
    \item[\texttt{prop\_65\_to\_74\_years}] \
          \begin{description}
              \item[Description] \agedescto{65}{74}
              \item[Reason for inclusion] \agereason
              \item[Data Source] \cite{acs_demographics_data}
              \item[Modifications] Changed from a percentage to a proportion.
          \end{description}
    \item[\texttt{prop\_75\_to\_84\_years}] \
          \begin{description}
              \item[Description] \agedescto{75}{84}
              \item[Reason for inclusion] \agereason
              \item[Data Source] \cite{acs_demographics_data}
              \item[Modifications] Changed from a percentage to a proportion.
          \end{description}
    \item[\texttt{prop\_85\_over\_years}] \
          \begin{description}
              \item[Description] \agedescover{85}
              \item[Reason for inclusion] \agereason
              \item[Data Source] \cite{acs_demographics_data}
              \item[Modifications] Changed from a percentage to a proportion.
          \end{description}
    \item[\texttt{prop\_18\_over\_years}] \
          \begin{description}
              \item[Description] \agedescover{18}
              \item[Reason for inclusion] \agereason
              \item[Data Source] \cite{acs_demographics_data}
              \item[Modifications] Changed from a percentage to a proportion.
          \end{description}
    \item[\texttt{prop\_21\_over\_years}] \
          \begin{description}
              \item[Description] \agedescover{21}
              \item[Reason for inclusion] \agereason
              \item[Data Source] \cite{acs_demographics_data}
              \item[Modifications] Changed from a percentage to a proportion.
          \end{description}
    \item[\texttt{prop\_62\_over\_years}] \
          \begin{description}
              \item[Description] \agedescover{62}
              \item[Reason for inclusion] \agereason
              \item[Data Source] \cite{acs_demographics_data}
              \item[Modifications] Changed from a percentage to a proportion.
          \end{description}
    \item[\texttt{prop\_65\_over\_years}] \
          \begin{description}
              \item[Description] \agedescover{65}
              \item[Reason for inclusion] \agereason
              \item[Data Source] \cite{acs_demographics_data}
              \item[Modifications] Changed from a percentage to a proportion.
          \end{description}
    \item[\texttt{median\_age}] \
          \begin{description}
              \item[Description] Median age of the population.
              \item[Reason for inclusion] \agereason
              \item[Data Source] \cite{acs_demographics_data}
              \item[Modifications] N/A
          \end{description}
    \item[\texttt{median\_household\_income}] \
          \begin{description}
              \item[Description] The median household income for the geographic area, in US dollars.
              \item[Reason for inclusion] This is a candidate for a predictor
                    variable. Median income is a better measure than mean for the
                    distribution of income when we care about the income of a
                    standard person living in an area, rather than a measure of
                    economic productivity.
              \item[Data Source] \cite{acs_income_data}
              \item[Modifications] N/A
          \end{description}
    \item[\texttt{prop\_households\_less\_than\_10000}] \
          \begin{description}
              \item[Description] The proportion of households making less than
                    \$10,000US.
              \item[Reason for inclusion] This is a potential predictor variable
                    to supplement and add detail to the number of people below the
                    poverty line.
              \item[Data Source] \cite{acs_income_data}
              \item[Modifications] Changed from a percentage to a proportion.
          \end{description}
    \item[\texttt{prop\_households\_10000\_to\_14999}] \
          \begin{description}
              \item[Description] The proportion of households making \$10,000US to
                    \$14,999US.
              \item[Reason for inclusion] This is a potential predictor variable
                    to supplement and add detail to the number of people below the
                    poverty line.
              \item[Data Source] \cite{acs_income_data}
              \item[Modifications] Changed from a percentage to a proportion.
          \end{description}
    \item[\texttt{prop\_25\_years\_over\_less\_than\_9th\_grade}] \
          \begin{description}
              \item[Description] \educationdesc{less than 9th grade}
              \item[Reason for inclusion] \educationreason
              \item[Data Source] \cite{acs_education_data}
              \item[Modifications] Changed from a percentage to a proportion.
          \end{description}
    \item[\texttt{prop\_25\_years\_over\_9th\_to\_12th\_no\_diploma}] \
          \begin{description}
              \item[Description] \educationdesc{9th to 12th grade, with no diploma}
              \item[Reason for inclusion] \educationreason
              \item[Data Source] \cite{acs_education_data}
              \item[Modifications] Changed from a percentage to a proportion.
          \end{description}
    \item[\texttt{prop\_25\_years\_over\_high\_school}] \
          \begin{description}
              \item[Description] \educationdesc{high school, with a diploma}
              \item[Reason for inclusion] \educationreason
              \item[Data Source] \cite{acs_education_data}
              \item[Modifications] Changed from a percentage to a proportion.
          \end{description}
    \item[\texttt{prop\_25\_years\_over\_some\_college}] \
          \begin{description}
              \item[Description] \educationdesc{some college}
              \item[Reason for inclusion] \educationreason
              \item[Data Source] \cite{acs_education_data}
              \item[Modifications] Changed from a percentage to a proportion.
          \end{description}
    \item[\texttt{prop\_25\_years\_over\_associates}] \
          \begin{description}
              \item[Description] \educationdesc{an associate's degree}
              \item[Reason for inclusion] \educationreason
              \item[Data Source] \cite{acs_education_data}
              \item[Modifications] Changed from a percentage to a proportion.
          \end{description}
    \item[\texttt{prop\_25\_years\_over\_bachelors}] \
          \begin{description}
              \item[Description] \educationdesc{a bachelor's degree}
              \item[Reason for inclusion] \educationreason
              \item[Data Source] \cite{acs_education_data}
              \item[Modifications] Changed from a percentage to a proportion.
          \end{description}
    \item[\texttt{prop\_married\_15\_years\_and\_older}] \
          \begin{description}
              \item[Description] \maritaldesc{married}
              \item[Reason for inclusion] \maritalreason
              \item[Data Source] \cite{acs_education_data}
              \item[Modifications] Changed from a percentage to a proportion.
          \end{description}
    \item[\texttt{prop\_widowed\_15\_years\_and\_older}] \
          \begin{description}
              \item[Description] \maritaldesc{widowed}
              \item[Reason for inclusion] \maritalreason
              \item[Data Source] \cite{acs_education_data}
              \item[Modifications] Changed from a percentage to a proportion.
          \end{description}
    \item[\texttt{prop\_divorced\_15\_years\_and\_older}] \
          \begin{description}
              \item[Description] \maritaldesc{divorced}
              \item[Reason for inclusion] \maritalreason
              \item[Data Source] \cite{acs_education_data}
              \item[Modifications] Changed from a percentage to a proportion.
          \end{description}
    \item[\texttt{prop\_separated\_15\_years\_and\_older}] \
          \begin{description}
              \item[Description] \maritaldesc{separated}
              \item[Reason for inclusion] \maritalreason
              \item[Data Source] \cite{acs_education_data}
              \item[Modifications] Changed from a percentage to a proportion.
          \end{description}
    \item[\texttt{prop\_never\_married\_15\_years\_and\_older}] \
          \begin{description}
              \item[Description] \maritaldesc{never married}
              \item[Reason for inclusion] \maritalreason
              \item[Data Source] \cite{acs_education_data}
              \item[Modifications] Changed from a percentage to a proportion.
          \end{description}
    \item[\texttt{prop\_below\_poverty\_level}] \
          \begin{description}
              \item[Description] Proportion of the population below the poverty level.
              \item[Reason for inclusion] This is a potential predictor
                    variable, providing more detail about poverty specifically
                    than median income.
              \item[Data Source] \cite{acs_poverty_data}
              \item[Modifications] Changed from a percentage to a proportion.
          \end{description}
\end{itemize}
\printbibliography

\end{document}
