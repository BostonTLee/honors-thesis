\documentclass{article}

\usepackage[backend=bibtex]{biblatex}

\usepackage[margin=1in]{geometry}

\usepackage{booktabs}
\usepackage[labelfont=bf]{caption}

\usepackage{enumitem}

\usepackage{graphicx}
\graphicspath{ {./images/} }

\usepackage{amsmath}
\usepackage{amsthm}
\usepackage{amssymb}

\usepackage[bookmarks, bookmarksopen=true, plainpages=false, pdfpagelabels,
pdfpagelayout=SinglePage, breaklinks = true, colorlinks=true, linkcolor=blue,
urlcolor=cyan]{hyperref}

\bibliography{references.bib}

\title{The Relationship Between Financial Stress and Mental Health in the United States}
\author{Boston Lee}

\begin{document}

\maketitle

\pagebreak

\section{Introduction}

Mental health issues are a growing societal and public health concern
in the United States.
There are a myriad of potential causes for the prevalence mental health
of mental health issues in the United States.
One plausible reason for mental health issues arising
on an individual or group level
is financial stress.

\section{Data Sources}

We obtained data from nationally representative sources
in the United States:
the American Community Survey (ACS),
the National Survey on Drug Use and Health (NSDUH).

The United States Census Bureau
conducts a yearly American Community Survey
to determine how to allocate federal
funding~\cite{acs_description}.
ACS estimates are obtained by
sending surveys to households
``randomly selected through a process of scientific sampling''
aimed at representing other similar households~\cite{acs_sampling}.


All ACS estimates were obtained from
the ACS 5-Year Estimates from the year 2018.

SAMHSA is an agency within
the U.S Department of Health and Human services which
``aims to reduce the impact of substance abuse and mental
illness on America's communities''~\cite{samhsa_about_us}.
SAMHSA conducts
the National Survey on Drug Use and Health (NSDUH),
which provides
``prevalence estimates for 15 measures of substance use and mental illness''
\cite{samhsa_data}.

The NSDUH estimates used in this paper are the
2016-2018 NSDUH Substate Region Estimates,
NSDUH data is collected via a multistage area probability sample,
in which regions are divided
into progressively smaller geographic units,
and interviews conducted with individuals at the smallest unit.
These interviews are aggregated to produce the NSDUH
estimates~\cite{nsduh_description}.

NSDUH provides definitions for substate regions in terms of
counties and census tracts for each collection term.
Each substate region is a collection of multiple
counties or census tracts contained within a single
given state~\cite{samhsa_substate_region_defs}.
The 2016-2018 substate region definitions were
pragmatically identical to the 2014-2016 substate region definitions,
and the 2014-2016 provided substate region definitions in a
usable data format, so the 2014-2016 substate region definitions
were used~\cite{samhsa_substate_region_defs}.

The main three response variables originated from the
SAMHSA data, while ACS provided demographic data used as controlling variables.

\subsection{Variables}\label{sec:finalvars}

The main response variables were chosen to measure mental health
symptoms that could have a relationship to environmental factors.
General mental illness, for instance, can be related to genetic factors
even more strongly than our predictor(s) of interest. The
mental-health-related variables chosen were: the proportion of individuals 12
or older who reported experiencing a major depressive episode in the
past year, the proportion of individuals 12 or older who reported
experiencing suicidal thoughts in the past year, and the proportion of
individuals 12 or older who reported an alcohol use disorder in the past
year. Each of these variables could intuitively have a relationship with
environmental variables, notably income, rather than genetic factors.

The main predictor variable, the (population weighted average of) the
proportion of individuals (FIXME) below the poverty level, was chosen
because we are attempting to discern the relationship between financial
stress and mental health. The proportion of individuals below the
poverty level encodes information about the lower tail of the
distribution of incomes, not simply the center of the distribution.
If a given region has more people living below the poverty line, we
would expect that to give us more information about financial stress
than simply a measure of centrality.

In addition, all variables are relative to time period.
See Section~\ref{sec:variables} for details.

A secondary question is whether the (population-weighted average of)
median income of a region does give us more information about mental
health outcomes, even given the proportion of individuals below the
poverty level. It is still feasible that a measure of center does encode
relevant information about financial stress, or is still related to
mental health outcomes. Median income was included rather than mean
income because mean income is sensitive to larger outliers, meaning in a
region where a proportion of the population is very wealthy, mean income
does not carry as much information about the income of a middle-income person
living in the region.

The model also included several demographic variables to control for
variability.

The proportion of population of Caucasian and African
American descent were chosen to control for potential socio-economic
effects on mental health outcomes. Other ethnicities were omitted to
ensure that no multicollinearity (FIXME) occured, given that knowing
multiple demographic variables gives information about what values other
proportions may be.

Median age was included as a controlling variable
both because median age could be related to mental health effects (in
that a region with a younger average population may experience more
reported or actual mental health problems in general) or income effects.

Education was accounted for by including the proportion of the
population in a region with a high school diploma or above. This
variable was included primarily to control for effects on income, given
that greater educational achievement is correlated with greater income.

Finally, the proportion of the population 15 years or older that is
married was included, to control for socioeconomic effects arising from
a greater marriage rate. For instance, reported household income is
likely higher in areas where more households consist of married
individuals (FIXME).

%
% county_fips
% substate_region_id
% substate_region_name
% state_fips
% state
% prop_MDE
% state_fips_x
% prop_suicidal_thoughts
% state_fips_y
% prop_alcohol_use_disorder
% total_pop
% prop_white
% prop_black
% median_age
% median_household_income
% prop_25_years_over_high_school
% prop_married_15_years_and_older
% prop_below_poverty_level
% fips
%

Note that in all of the below descriptions, the variables are described in their
non-aggregated state. A full description of the census-level variables would
include a caveat that the ACS variables are actually weighted averages of
county-level estimates. For more information on variable aggregation, see
Section~\ref{sec:aggregation}. Treat references to, for instance, the
median income, as references to the population-weighted average of
median income across a given substate region.


For a full description of variables, including data sources, see
Appendix, Section \ref{sec:variables}.

\subsection{Data Joining}\label{sec:joining}

The dataset was formed with substate regions corresponding to
``substate/aggregate region-level estimates included in the [SAMHSA] maps''
\cite{samhsa_data}. All substate regions appearing in datasets with the response
variables of interest were used as individual data points in the final dataset.
However, the final dataset also included ACS county-level estimates. As such,
the SAMHSA estimates were duplicated for every county belonging to the region,
and then aggregated back to substate-region estimates (see Section
\ref{sec:aggregation}).
For census-tract-level data, the ACS estimates
were applied for the county to which the census tract belongs.

\subsection{Data Aggregation}\label{sec:aggregation}

All variables in the model dataset were aggregated to the substate region level,
as defined by the SAMHSA data \cite{samhsa_substate_region_defs}.
The ACS estimates were only available at the county level,
while the response variables were only available at the substate region level.
Hence, while the preprocessed and joined data was
initially represented at the county level,
with repeated observations for the SAMHSA estimates
the final models were constructed at the substate region level.
That is, the final models were fit using data at the original
scale provided by the SAMHSA estimates, before the joining described in
Section~\ref{sec:joining}.
The ACS estimates were collapsed to the substate region level
by a population weighted average for the substate region in question.
Each numeric variable was grouped by substate region
in the following way:
Assume group $g$ contains dataset indices $I_g$,
each of which is associated with a county-level population
$p_{i}, i \in I_g$.
Then given a set of observations $x_{i}, i \in I_g$
from a numeric variable,
the condensed estimate $x_g$ would be as follows:

\begin{equation*}
    x_g = \sum_{i \in I_g}
    \left[ \frac{p_{i}}{\sum_{k \in I_g} p_{k}} x_i \right]
\end{equation*}

Note that because of the joining procedure,
we want the SAMHSA variables measured at the substate regional level,
that is, the level to which we are aggregating,
to remain the same as before the joining procedure.
We can see that the above approach accomplishes this.
If all $x_i, i \in I_g$ are equal to $\xi$,
then our procedure reduces to the following:

\begin{equation*}
    x_g =
    \sum_{i \in I_g}
    \left[ \frac{p_{i}}{\sum_{k \in I_g} p_{k}} x_i \right] =
    \sum_{i \in I_g}
    \left[ \frac{p_{i}}{\sum_{k \in I_g} p_{k}} \xi \right] =
    \left[ \frac{\sum_{i \in I_g}p_{i}}{\sum_{k \in I_g} p_{k}} \right] \xi =
    \xi
\end{equation*}

This type of aggregation thus has an asymmetrical (albeit desirable)
effect on the variables from different sources.
It leaves the substate-region-level SAMHSA estimates unchanged,
while providing substate-region-level ACS aggregated estimates
that reflect the weight of relative population within a given substate region.

After the county-level population estimates were used
to create a weighted average for the numeric variables,
they were dropped from the final dataset.
That is, population was not used as a predictor or response variable.
Only the variables mentioned in \ref{sec:finalvars}
were included in the final model.

\section{Descriptive Statistics}

As seen in Table~\ref{tab:summary},
most variables in the dataset were proportions restricted to $[0, 1]$.
However, many of the proportion variables took values on a much cmaller range.
Proportion of individuals experiencing a major depressive episode
(\texttt{prop\_MDE}),
for instance, achieved a maximum of 0.102.
The other two SAMHSA NSDUH variables achieved maxima under 0.2.
As a result, these variables had very low standard deviation.
The proportion of individuals under the poverty line was similarly
restricted to a narrower range.
Median age had very low deviation compared to the mean value.

The distributional characteristics of the proportion of African Americans
in a region showed significant skew.
Although the median proportion of African Americans in a region was 0.06,
the maximum was 0.63, and the mean was 0.111.


\begin{table}[!htb]
\begin{center}
\begin{tabular}{l r r r r r}
    \toprule
    & Mean & SD & Min & Median & Max\\
    \midrule
    prop\_MDE & 0.073 & 0.009 & 0.050 & 0.073 & 0.102\\
    prop\_suicidal\_thoughts & 0.045 & 0.006 & 0.030 & 0.044 & 0.071\\
    prop\_alcohol\_use\_disorder & 0.057 & 0.012 & 0.035 & 0.055 & 0.119\\
    prop\_white & 0.766 & 0.162 & 0.180 & 0.815 & 0.971\\
    prop\_black & 0.111 & 0.128 & 0.001 & 0.060 & 0.630\\
    median\_age & 38.792 & 3.650 & 24.600 & 38.839 & 51.431\\
    median\_household\_income & 59911.078 & 14078.759 & 32334.991 & 56552.785 & 117265.848\\
    prop\_25\_years\_over\_high\_school & 0.289 & 0.063 & 0.123 & 0.292 & 0.466\\
    prop\_married\_15\_years\_and\_older & 0.485 & 0.057 & 0.269 & 0.492 & 0.613\\
    prop\_below\_poverty\_level & 0.143 & 0.043 & 0.054 & 0.141 & 0.305\\
    \bottomrule
\end{tabular}
\end{center}
\caption{\label{tab:summary}Measures of distribution of
all numeric variables included in the final dataset used for modeling.
This includes median income, although median income was not included
in the primary model, and only included in a sensitivity model.
Variable names are given as they were coded in the model.
For details, see Section~\ref{sec:variables}.}
\end{table}

% \begin{figure}[!htb]
    % \centering
    % \includegraphics[width=0.75\textwidth]{}
    % \caption{a nice plot}
    % \label{fig:}
% \end{figure}

\begin{figure}[!htb]
    \centering
    \includegraphics[width=0.75\textwidth]{prop_below_poverty_level_aggregated.png}
    \caption{A map of the population-weighted average of the proportion of
    people living below poverty in each substate region.
	For more information about how this data was aggregated,
	see Section~\ref{sec:aggregation}.
	}
    \label{fig:map-poverty}
\end{figure}

\begin{figure}[!htb]
    \centering
    \includegraphics[width=0.75\textwidth]{prop_MDE_aggregated.png}
    \caption{A map of the proportion of individuals 12 and older who
	reported having a Major Depressive Episode during the last 12 months.}
    \label{fig:map-MDE}
\end{figure}

\begin{figure}[!htb]
    \centering
    \includegraphics[width=0.75\textwidth]{prop_suicidal_thoughts_aggregated.png}
    \caption{A map of the proportion of individuals 12 and older who
	reported having suicidal thoughts during the last 12 months.}
    \label{fig:map-suicidal-thoughts}
\end{figure}

\begin{figure}[!htb]
    \centering
    \includegraphics[width=0.75\textwidth]{prop_alcohol_use_disorder_aggregated.png}
    \caption{A map of the proportion of individuals 12 and older who
	reported having an alcohol use disorder during the last 12 months.}
    \label{fig:map-alcohol-use}
\end{figure}


\section{Methods}

\subsection{Primary Model Group}

The primary models used were linear regression models with the three SAMHSA
variables as response variables, and
(the population-weighted average of)
proportion of individuals below the poverty line as a primary predictor.
The other demographic variables were included as control variables,
and
(the population-weighted average of)
median household income was not included.

These three models were treated as the primary test
of a significant relationship,
and as such were the only models to which a Bonferroni-Holm
$p$-value adjustment was performed \cite{holm}.
For tests, interest was in the relationship between
(the population-weighted average of)
proportion of individuals below the poverty line
and each of the three SAMHSA variables.
The other controlling variables were not
observed for significance,
and thus not considered in the testing adjustment.


\subsection{Median Income Model Group}

The second group of three linear regression models included all variables
from the primary model,
and
(the population-weighted average of)
median household income.
These three models were fit to ascertain whether
the relationship between the
(the population-weighted average of)
proportion of individuals below the poverty line
and the SAMHSA variables was altered in the
with the inclusion of a measure of center for income.

\subsection{State Indicator Model Group}

The third group of three linear regression models included all variables
from the primary model,
and and indicator for US state for each the
substate regions defined in the SAMHSA data.
State membership was a variable of interest for two
reasons:
First, the visualizations of the predictor and response variables
showed that if there was evidence of spatial variation,
that spatial variation was strongest within state boundaries
(see Figures
\ref{fig:map-poverty},
\ref{fig:map-MDE},
\ref{fig:map-suicidal-thoughts},
\ref{fig:map-alcohol-use}
).
Second, although there did not seem to be string spatial
clustering in the fitted values of the primary model group,
it was still of interest whether adding a spatial component to
the primary model would alter any relationship between
the predictor and response for each model.
Adding an indicator for state helped accomplish both goals.
Furthermore, adding an indicator for state was much less
time-intensive than attempting to form a precise spatial model
that captured all the specially-defined SAMHSA substate regions.


\section{Results}

\subsection{Primary Model Group}

As can be seen from Table~\ref{tab:primary-model-results},
after adjustment,
the coefficient for the percentage of people below the poverty level
was only significantly different from zero
in the model with proportion of people reporting
alcohol use disorder as a response.

We can see that the (unadjusted) confidence intervals for both
the proportion of people reporting a MDE
and the proportion of people reporting suicidal thoughts
are relatively narrow around zero.

\begin{table}[!htb]
\begin{center}
    \begin{tabular}{lrrl}
        \toprule
        Response variable & Coefficient & $p$-value & 95\% CI\\
        \midrule
        Major Depressive Episode (MDE) & -0.024 & 0.155 & (-0.05, 0.003)\\
        Suicidal Thoughts & -0.003 & 0.729 & (-0.022, 0.015)\\
        Alcohol Use Disorder & -0.086 & 1.0e-7 & (-0.116, -0.056)\\
        \bottomrule
    \end{tabular}
\caption{\label{tab:primary-model-results} The coefficients and $p$-values
    for the main predictor for each of the three models
    in the primary model group.
}
\end{center}
\end{table}

\begin{figure}[!htb]
    \centering
    \includegraphics[width=0.75\textwidth]{prop_MDE_fitted.png}
    \caption{A map of the fitted values for proportion of people experiencing an
    MDE, from the primary model group. For the origial values, see
    Figure~\ref{fig:map-MDE}.}
    \label{fig:map-fitted-MDE}
\end{figure}

\begin{figure}[!htb]
    \centering
    \includegraphics[width=0.75\textwidth]{prop_suicidal_thoughts_fitted.png}
    \caption{A map of the fitted values for proportion of people experiencing
    suicidal thoughts, from the primary model group. For the origial values, see
    Figure~\ref{fig:map-suicidal-thoughts}.}
    \label{fig:map-fitted-suicidal-thoughts}
\end{figure}

\begin{figure}[!htb]
    \centering
    \includegraphics[width=0.75\textwidth]{prop_alcohol_use_disorder_fitted.png}
    \caption{A map of the fitted values for proportion of people experiencing an
    alcohol use disorder, from the primary model group.
    For the origial values, see Figure~\ref{fig:map-alcohol-use}.}
    \label{fig:map-fitted-alcohol-use}
\end{figure}

One concern with using linear models was extrapolation
beyond the bounds of a proportion.
The fitted values for each region can be seen in
Figures~\ref{fig:map-fitted-MDE},\
~\ref{fig:map-fitted-suicidal-thoughts},\
~and~\ref{fig:map-fitted-alcohol-use}.
None of the fitted values were outside of $[0,1]$,
meaning there was no structural violation of the
variables in the fitted values.
A table of the minimum and maximum fitted values
attained by each model is found in the Appendix,
Table~\ref{tab:fitted}.


\subsection{Median Income Model Group}

\begin{table}[!htb]
\begin{center}
\begin{tabular}{l r r l}
    \toprule
    Response variable & Coefficient & $p$-value & 95\% CI\\
    \midrule
    Major Depressive Episode (MDE) & -0.027 & 0.218 & (-0.07, 0.016)\\
    Suicidal Thoughts & -0.030 & 0.046 & (-0.06, -0.001)\\
    Alcohol Use Disorder & -0.068 & 0.006 & (-0.116, -0.019)\\
    \bottomrule
\end{tabular}
\caption{\label{tab:median-income-model-results} The coefficients and $p$-values
    for the main predictor for each of the three models
    in the median-income-adjusted model group.
}
\end{center}
\end{table}

As seen in Table~\label{tab:median-income-model-results},
ompared to the primary model the model adjusted for median income
was not drastically different with respect to coefficient values and
confidence interval bounds.

The coefficient for the proportion of people under
poverty in the model with suicidal thoughts as the response
was an order of magnitude larger in the median-income-adjusted model,
with an overall difference in the coefficient was only 0.027.

% FIXME
The right bound of the CI for $\hat{\beta}_{pov}$ in the
model with suicidal thoughts as the response
shifted to the other side of zero than the corresponding model
in the primary model group.
However, the right bound was very close to zero,
with a value of -0.001.


\subsection{State Indicator Model Group}

As seen in Table~\label{tab:state-ind-model-results},
all three of the confidence intervals for
$\hat{\beta}_{pov}$ contained zero.

The coefficient $\hat{\beta}_{pov}$ was smaller
in the state-adjusted model compared to the
corresponding model in the
primary model group.
The other two coefficients were close to their
primary model counterparts.

\begin{table}[!htb]
\begin{center}
\begin{tabular}{l r r l}
    \toprule
    Response variable & Coefficient & $p$-value & 95\% CI\\
    \midrule
    Major Depressive Episode (MDE) & -0.022 & 0.093 & (-0.047, 0.004)\\
    Suicidal Thoughts & 0.001 & 0.944 & (-0.014, 0.015)\\
    Alcohol Use Disorder & -0.020 & 0.176 & (-0.05, 0.009)\\
    \bottomrule
\end{tabular}
\caption{\label{tab:state-ind-model-results} The coefficients and $p$-values
    for the main predictor for each of the three models
    in the state-indicator-adjusted model group.
}
\end{center}
\end{table}

\subsection{Only State Indicator Model Group}


As seen in Table~\label{tab:state-ind-only-model-results},
the confidence intervals for $\hat{\beta}_{pov}$
in the models with the proportion of people
reporting suicidal thoughts
and the proportion of people reporting an alcohol use disorder
both did not contain zero.
In addition, all values of $\hat{\beta}_{pov}$
were positive in this model group,
while in the primary model group,
the values of $\hat{\beta}_{pov}$
were all negative.
This model group was the most different from the primary model group,
given that the coefficients of the models had the opposite sign,
and that $\hat{\beta}_{pov}$ in the model
with the proportion of people reporting suicidal thoughts
as the response had a CI that did not contain zero.

\begin{table}[!htb]
\begin{center}
\begin{tabular}{l r r l}
    \toprule
    Response variable & Coefficient & $p$-value & 95\% CI\\
    \midrule
    Major Depressive Episode (MDE) & 0.004 & 0.680 & (-0.014, 0.022)\\
    Suicidal Thoughts & 0.019 & 4.33e-4 & (0.008, 0.029)\\
    Alcohol Use Disorder & 0.021 & 0.045 & (0.001, 0.042)\\
    \bottomrule
\end{tabular}
\caption{\label{tab:state-ind-only-model-results} The coefficients and $p$-values
    for the main predictor for each of the three models
    in the state-indicator-adjusted model group with no controlling demographic variables.
}
\end{center}
\end{table}

\subsection{Response Variable Correlations}~\label{sec:results-corr}

The correlation between the proportion of people who reported experiencing
a major depressive episode and
who reported experiencing suicidal thoughts was
0.765.
This is intuitive; we would expect depressive symptoms and suicidal thoughts
to be positively correlated.
Furthermore, this correlation is fairly strong.

The correlation between the proportion of people who reported experiencing
a major depressive episode and
who reported experiencing an alcohol use disorder was
0.224.
This correlation is positive, but significantly weaker than
the previous correlation.
The correlation between the proportion of people who reported experiencing
suicidal thoughts and
who reported experiencing an alcohol use disorder was
0.162.
Again, this correlation is positive, but still weaker than
both the previous correlations.

The results from these correlations show that the proportion individuals
experiencing MDE and suicidal thoughts are much more strongly related to each
other than either of them are to the proportion of people experiencing
an alcohol use disorder.
Of the three responses, the proportion of people experiencing an
alcohol use disorder is the least specific.


\section{Discussion}

% It should also be noted that because data is aggregated to regional levels, we
% cannot infer a relationship between financial variables and mental health
% variables on an individual level, but rather on a regional level. We may be
% able to say that, for instance, a difference of $x$ in the average proportion
% of individuals living below poverty in a region is correlated to a difference
% of $y$ in the average proportion of individuals experiencing a major depressive
% episode. We may \textbf{not} say, given our data, that a difference of $x$ in
% an individual's average income is correlated to a difference of $y$ in the
% average of an individual's probability of experiencing a major depressive
% episode.

If it were discovered that financial hardship and mental
health issues covary, it would perhaps inform
either mental health policy, economic policy, or both.
Although the standard statistical methods used in this paper
cannot be used to draw causal conclusions with observational data,
perhaps insight into whether there is evidence that there
is a linear relationship between poverty and negative mental health effects
could be used to guide more sophisticated future research.

The most interesting result from the primary model group
was the fact that of the three primary models,
only the model with reported alcohol use disorder
as the predictor had a significant $\hat{\beta}_{pov}$
(see Table \ref{tab:primary-model-results}).
The two response variables in the models in which
$\hat{\beta}_{pov}$ was not sigificant were also
seen to be much more strongly correlated to each other
than either was to reported alcohol use disorder
(see Section~\ref{sec:results-corr}).
This shows that alcohol use disorder may not
be a good proxy for mental health.
Intuitively, reported MDEs and suicidal thoughts
are signals of strong negative mental health events,
and both seem more strongly indicative of negative mental health events
than alcohol use disorder.
This is bourne out by the fact that reported MDEs and suicidal thoughts
are correlated relatively strongly.
While alcohol use disorder may result from the type of negative
mental stress that this paper aims to study,
alcohol use may also be attributed to a number
of other personal factors, some of which are not conventionally
negative at all.
For instance, excess drinking at social events or parties
may be classified as an alcohol use disorder if
an individual has trouble controlling their drinking.
Overall, it seems that alcohol use,
suffers from similar problems we would expect in attempting
to other other substance abuse problems
as a proxy for mental health:
Substance abuse problems can arise from many other
factors than specifically environmental stressors.

The considerations above give an explanation for the fact that
not only did only the alcohol use disorder model have a
significant $\hat{\beta}_{pov}$, the value of $\hat{\beta}_{pov}$
in that model was negative.
That is, we would associate a positive difference in
the proportion of the population living under the poverty line
(ie, more people in poverty)
with a negative difference in the prevalence of alcohol use disorder,
According to the coefficient value from our model, presented in
Table \ref{tab:primary-model-results}
(See Table~\ref{tab:primary-alcohol-use-full-summary} for a full
summary of the model in question).
The fact that the direction of this coefficient is opposite of
what we would expect given the subject matter
indicates there are likely mediating variables in this relationship.
For instance, the price of alcohol may mean those living below
the poverty line are less inclined to buy alcohol,
and thus less likely to experience an alcohol use disorder.
It may be the result for the model with alcohol use as
a predictor is genuinely reflective of the relationship
we intended to study, but there is little transparency
as to what mediating effects may exist,
and how then we might interpret the significance of
the coefficient for this particular model.

A second concern was that of spatial correlation.
There was no evidence of spatial correlation in the residuals
of the primary models;
however, the resesarch plan laid out the state-indicator
models to investigate whether a proxy for geographic
location would affect the significance of
$\hat{\beta}_{pov}$
in any of the models.
The state-indicator models including demographic variables
did not have any models whose 95\% CIs for
$\hat{\beta}_{pov}$
did not include zero
(see Table~\ref{state-ind-model-results}).
That is, the addition of state indicator variables did
not seem to cause the model to detect a non-zero effect,
for any of the response variables considered,
including alcohol use disorder, which did have a
CI that did not include zero in the primary model.
This could be a function of the fact that adding
state indicators adds fifty additional predictor
variables, causing a loss of power.
The second state-indicator model group
(see Table~\ref{state-ind-only-model-results})
which did \textit{not} adjust with any demographic variables,
did have two confidence intervals not containing zero.
This could be because there is spatial variation
in the two responses whose CIs did not contain zero
(reported suicidal thoughts and reported alcohol use disorder).
However, this could have also been because there is
spatial variation in an important demographic covariate of
these two responses, causing the appearance of spatial variation.

The considerations above are complicated by the fact that
state membership is a rather crude proxy for the spatial relationships
between substate regions.
Due to time constraints, this paper did not include such spatial measures,
because the NSDUH substate region definitions
are not a standard geographic unit,
and thus the spatial relationships between substate regions
would have to be encoded manually.
This paper opted for an alternative approach of
providing some indication of region location (state membership)
which did not require finding the geographic centers of
and distances between the custom NSDUH substate regions.
In future research,
a more thorough examination of spatial relationships
could take distances between substate regions into consideration,
providing a fuller picture of the actual relative location
of each region.
A spatial model may reveal a clearer picture of the relationship
between the predictor and responses
than the linear models in this paper.
Future research would be made easier if estimates for
the response variables of interest---either those
used in this paper, or more suited ones as mentioned
later in the Discussion---were available
at a more standard geographic scale
(say, at the county level).

Note that given the unavailability of individual-level data,
and the implausibility of conducting randomized trials with our chosen responses,
some compromises were made in the research questions explored in this paper.
Data for our responses and predictors was only available on a regional level,
hence any inferential conclusions drawn
concern geographic regions, not individuals.
Any conclusions about group-level data must be taken very tentatively,
to avoid commiting an ``ecological fallacy'',
wherein one attempts to erroneously extend group-level inferences
to individuals
\cite{piantadosi_1988}.
While this paper may attempt to describe
intuition for causes of certain behaviors on an individual
level, we hope to avoid making inference at the individual level.

Furthermore, any inferential conclusions drawn will concern correlative
relationships.
Data about prevalence of mental health issues and poverty is observational
in nature, and thus modeling these phenomena does not license causal
conclusions.

Even barring problems with ecological bias,
the results from all of the models presented in this paper
are somewhat difficult to interpret,
given that it is dubious whether the response variables
used in each model are at all representative
of how financial stress might be exhibited
on either a group or individual level.
In lieu of variables about mental health or substance abuse,
a better measure may be overall reported stress levels.
Instead of framing the research question as investigating
the link between financial stress and mental health,
it would be perhaps more fruitful to investigate
the link between financial hardship
and overall stress.
That is, instead of assuming poverty status
is indicative of financial stress,
it would be better to find out to what extent
financial hardship is indicative of stress
in the first place.

Furthermore, although stress is likely influenced by
extraneous genetic and environmental factors,
stress on an individual level is potentially more
senstitive to general lifestyle conditions than
, for instance,
major depressive episodes.

A different approach along the same lines would be
to measure reported financial stress directly,
and---with this more accurate predictor in hand---again
investigate the link between financial stress
and mental health.
Not only may financial stress be poorly linked to
the specific mental health effects used in this paper,
it is possible that financial stress is poorly linked
to nominal ``financial hardship''.
If we are interested in the link between financial stress and mental health,
there may be no good widely-available proxy for financial stress.

Both of the complications mentioned above---either
that financial hardship is is linked to different mental
health effects than the ones used here,
or that financial hardship is a poor proxy for financial
stress---individually or jointly account for the
failure of the primary models to detect a relationship
between the proportion of the population below the poverty line
and either of the two strictly mental health effects,
reported MDEs and suicidal thoughts.

\section{Appendix}

The code used to clean data and fit the models for this
paper, as well as the raw document itself,
is publicly available at
\url{https://github.com/BostonTLee/honors-thesis}.

All data cleaning and joining was done with the \texttt{Pandas} Python 3 package
\cite{python_pandas}, as well as the package \texttt{python-us}~\cite{python_us}.
Of course, this required the use of Python 3 programming language
\cite{python_lang}.

All model creation was done using the \texttt{R} programming
language, version 3.6.3~\cite{r_lang}.
All data aggregation and visualization was done with the
\texttt{tidyverse} package in \texttt{R}~\cite{r_tidyverse}.

\subsection{Variable Descriptions}\label{sec:variables}

\newcommand\censuscodes{https://www.census.gov/programs-surveys/acs/technical-documentation/code-lists.2018.html}
\newcommand\descentdesc[1]{ Proportion of the population which is of
    #1 descent. See \url{\censuscodes} for more
    pecise definitions.}
\newcommand\descentreason{Ethnicity is tied to socioeconomic status, which
    could in turn be correlated to one's financial status and mental health.
    Variables related to ethnicity were included to control for variation.}
\newcommand\agedescto[2]{Proportion of the population from #1 to #2 years old.}
\newcommand\agedescover[1]{Proportion of the population over #1 years old.}
\newcommand\agereason{Overall economic status could vary with age, so age
    variables were included to control for variation.}
\newcommand\educationdesc[1]{Proportion of the population 25 years of age and
    over whose maximum educational attainment is #1.}
\newcommand\educationreason{Educational attainment can be used as a general
    measure of socioeconomic status. Variables related to educational
    attainment were included to account for variation.}
\newcommand\maritaldesc[1]{Proportion of the population age 15 and older who
    have marital status ``#1".}
\newcommand\maritalreason{Marital status reflects general socioeconomic status.
    Variables related to marital status are included to account for variation.}

\begin{itemize}[label={}, align=left]
    \item[\texttt{county\_fips}] \
          \begin{description}
              \item[Description] FIPS code at the county level (three digits)
              \item[Reason for inclusion] Used as a merge key for all datasets.
                    Extracted from longer FIPS codes as needed. \\
              \item[Data Source]
                    \cite{acs_demographics_data},
                    \cite{acs_poverty_data},
                    \cite{acs_income_data},
                    \cite{acs_marital_data},
                    \cite{acs_education_data},
                    \cite{samhsa_data} \\
              \item[Modifications] County-level FIPS was defined within a
                    larger string. This string was sliced and zero-padded for all
                    datasets to produce a consistent key. \\
          \end{description}
    \item[\texttt{substate\_region\_id}] \
          \begin{description}
              \item[Description] A numeric value that corresponds to the
                    substate region name within a state. If a state has "n"
                    substate regions, then this variable will take on values 1,
                    2, ..., n, within that state.
              \item[Reason for inclusion] This, in combination with state ID,
                    gives a unique identifier for the substate regions used in the
                    SAMHSA data.
              \item[Data Source] \cite{samhsa_substate_region_defs}
              \item[Modifications] N/A
          \end{description}
    \item[\texttt{substate\_region\_name}] \
          \begin{description}
              \item[Description] The substate region name used in the 2014-2016
                    substate report, attached at

                    \parbox[t]{\textwidth}{\citefield{samhsa_substate_region_defs}{url}}

              \item[Reason for inclusion] Human-readable name for substate
                    regions.
              \item[Data Source] \cite{samhsa_substate_region_defs}
              \item[Modifications] N/A
          \end{description}
    \item[\texttt{state\_fips}] \
          \begin{description}
              \item[Description] FIPS code at the state level (two digits) \\
              \item[Reason for inclusion] Used as a merge key for all datasets.
                    Extracted from longer FIPS codes as needed. \\
              \item[Data Source]
                    \cite{acs_demographics_data},
                    \cite{acs_poverty_data},
                    \cite{acs_income_data},
                    \cite{acs_marital_data},
                    \cite{acs_education_data},
                    \cite{samhsa_data} \\
              \item[Modifications] State-level FIPS was defined within a
                    larger string. This string was sliced and zero-padded for all
                    datasets to produce a consistent key. \\
          \end{description}
    \item[\texttt{census\_tract\_code}] \
          \begin{description}
              \item[Description] 2010 Geography census tract code (where
                    applicable).
              \item[Reason for inclusion] Used for identifying tracts belonging
                    to a substate region defined at the census tract level.
              \item[Data Source] \cite{samhsa_substate_region_defs}
              \item[Modifications] N/A
          \end{description}
    \item[\texttt{prop\_MDE}] \
          \begin{description}
              \item[Description] Proportion of adults aged 18 or older major
                    who have had a major depressive episode in the past year
              \item[Reason for inclusion] This is a candidate for a response
                    variable. Having a major depressive episode(s) is more
                    closely linked to life events than general mental illness, in
                    which there is a lot of variation.
              \item[Data Source] \cite{samhsa_data}
              \item[Modifications] Changed from a percentage to a proportion.
          \end{description}
    \item[\texttt{prop\_suidical\_thoughts}] \
          \begin{description}
              \item[Description] Proportion of individuals aged 12 or older
                    with an alcohol use disorder in the past year \\
              \item[Reason for inclusion] This is a potential response variable
                    to measure overall mental health in a population. Suicidal
                    thoughts in particular seem to be more closely linked to life
                    circumstances than genetic factors, as opposed to general
                    mental illness.
              \item[Data Source] \cite{samhsa_data} \\
              \item[Modifications] Changed from a percentage to a proportion.
          \end{description}
    \item[\texttt{prop\_white}] \
          \begin{description}
              \item[Description Auto] \descentdesc{Caucasian}
              \item[Reason for inclusion] \descentreason
              \item[Data Source] \cite{acs_demographics_data}
              \item[Modifications] Changed from a percentage to a proportion.
          \end{description}
    \item[\texttt{prop\_black}] \
          \begin{description}
              \item[Description] \descentdesc{African American}
              \item[Reason for inclusion] \descentreason
              \item[Data Source] \cite{acs_demographics_data}
              \item[Modifications] Changed from a percentage to a proportion.
          \end{description}
    \item[\texttt{median\_age}] \
          \begin{description}
              \item[Description] Median age of the population.
              \item[Reason for inclusion] \agereason
              \item[Data Source] \cite{acs_demographics_data}
              \item[Modifications] N/A
          \end{description}
    \item[\texttt{median\_household\_income}] \
          \begin{description}
              \item[Description] The median household income for the geographic area, in US dollars.
              \item[Reason for inclusion] This is a candidate for a predictor
                    variable. Median income is a better measure than mean for the
                    distribution of income when we care about the income of a
                    standard person living in an area, rather than a measure of
                    economic productivity.
              \item[Data Source] \cite{acs_income_data}
              \item[Modifications] N/A
          \end{description}
    \item[\texttt{prop\_25\_years\_over\_high\_school}] \
          \begin{description}
              \item[Description] \educationdesc{high school, with a diploma}
              \item[Reason for inclusion] \educationreason
              \item[Data Source] \cite{acs_education_data}
              \item[Modifications] Changed from a percentage to a proportion.
          \end{description}
    \item[\texttt{prop\_married\_15\_years\_and\_older}] \
          \begin{description}
              \item[Description] \maritaldesc{married}
              \item[Reason for inclusion] \maritalreason
              \item[Data Source] \cite{acs_education_data}
              \item[Modifications] Changed from a percentage to a proportion.
          \end{description}
    \item[\texttt{prop\_below\_poverty\_level}] \
          \begin{description}
              \item[Description] Proportion of the population below the poverty level.
              \item[Reason for inclusion] This is a potential predictor
                    variable, providing more detail about poverty specifically
                    than median income.
              \item[Data Source] \cite{acs_poverty_data}
              \item[Modifications] Changed from a percentage to a proportion.
          \end{description}
\end{itemize}

\pagebreak

\subsection{Tables}


\begin{table}[!htb]
\begin{center}
\begin{tabular}{lll}
    \toprule
    Variable & Estimate & 95\% CI\\
    \midrule
    (Intercept) & 0.09913 & (0.083, 0.115)\\
    prop\_white & 0.03951 & (0.031, 0.048)\\
    prop\_black & 0.0044 & (-0.006, 0.015)\\
    median\_age & -0.00086 & (-0.001, -0.001)\\
    prop\_25\_years\_over\_high\_school & 0.01934 & (0.001, 0.037)\\
    prop\_married\_15\_years\_and\_older & -0.053 & (-0.078, -0.028)\\
    prop\_below\_poverty\_level & -0.02371 & (-0.05, 0.003)\\
    \bottomrule
\end{tabular}
\end{center}
\caption{\label{tab:primary-mde-full-summary}
    Full model summary for the primary model with
    proportion of MDE
    as the response.
}
\end{table}

\begin{table}[!htb]
\begin{center}
    \begin{tabular}{lll}
    \toprule
    Variable & Estimate & 95\% CI\\
    \midrule
    (Intercept) & 0.06173 & (0.051, 0.073)\\
    prop\_white & 0.01665 & (0.011, 0.023)\\
    prop\_black & -0.00443 & (-0.012, 0.003)\\
    median\_age & -0.00075 & (-0.001, -0.001)\\
    prop\_25\_years\_over\_high\_school & 0.01301 & (0, 0.026)\\
    prop\_married\_15\_years\_and\_older & -0.00745 & (-0.025, 0.01)\\
    prop\_below\_poverty\_level & -0.00324 & (-0.022, 0.015)\\
    \bottomrule
\end{tabular}
\end{center}
\caption{\label{tab:primary-suicidal-thoughts-full-summary}
    Full model summary for the primary model with
    reported proportion of suicidal thoughts
    as the response.
}
\end{table}

\begin{table}[!htb]
\begin{center}
    \begin{tabular}{lll}
        \toprule
        Variable & Estimate & 95\% CI\\
        \midrule
        (Intercept) & 0.16519 & (0.147, 0.184)\\
        prop\_white & 0.02282 & (0.013, 0.033)\\
        prop\_black & -0.02457 & (-0.036, -0.013)\\
        median\_age & -0.00021 & (-0.001, 0)\\
        prop\_25\_years\_over\_high\_school & -0.04412 & (-0.065, -0.024)\\
        prop\_married\_15\_years\_and\_older & -0.18505 & (-0.213, -0.157)\\
        prop\_below\_poverty\_level & -0.08641 & (-0.116, -0.056)\\
        \bottomrule
    \end{tabular}
\end{center}
\caption{\label{tab:primary-alcohol-use-full-summary}
    Full model summary for the primary model with
    reported proportion of alcohol use disorder
    as the response.
}
\end{table}


\begin{table}[!htb]
\begin{center}
\begin{tabular}{l l l}
    \toprule
    Response Variable & Minimum fitted value & Maximum fitted value\\
    \midrule
    prop\_MDE & 0.052 & 0.085\\
    prop\_suicidal\_thoughts & 0.034 & 0.056\\
    prop\_alcohol\_use\_disorder & 0.037 & 0.082\\
    \bottomrule
\end{tabular}
\end{center}
\caption{\label{tab:fitted} Minimum and maximum fitted values
    for the primary model group. For variable descriptions,
    see Section~\ref{sec:variables}.
}
\end{table}

\pagebreak

\printbibliography

\end{document}
